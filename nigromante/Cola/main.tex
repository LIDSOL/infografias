\documentclass{article}
\usepackage[utf8]{inputenc}


\title{Mi primer año con software libre}
\author{Pablo Vivar Colina}
\date{Octubre 2017}

%IDIOMA
\usepackage[spanish.mexico]{babel}

\usepackage{natbib}
\usepackage{graphicx}

%ELEGANTE
\usepackage{fancyhdr}
\pagestyle{fancy}



\begin{document}

\maketitle



%\section{Introducción}

Cuando estás por entrar a la universidad varias preguntas se hacen importantes en el momento de comprar una computadora, si puede ejecutar el programa que necesitas sin errores, si te va a durar toda la carrera, si tiene Office completo, etc.\\

Son preguntas simples, que puedes resolver con relativa facilidad en internet, pero la pregunta correcta sería ¿En realidad necesitamos todo lo que las empresas de la bandera multicolor y la manzana nos %compañías de computación y software nos 
venden?\\

 Como muchos al entrar en la carrera tuve las mismas preguntas, y las respondí de la manera tradicional: Compré una computadora que tenía todo lo que creía que necesitaba, y eso implicaba correr AutoCAD.\\
 
 Al pasar tiempo en la carrera, un Amigo, Sebastián Aguilar, me habló sobre el software libre y sus ventajas, y porqué las personas deberíamos usarlos en vez de los programas con licencia. Al inicio, seguí usando el software como acostumbraba, pero poco a poco me fuí adentrando al mundo del software libre. Comencé instalando la distribución del sistema GNU/Linux Ubuntu en mi computadora de escritorio, y experimenté la experiencia de tener un sistema operativo libre.\\
 
 Al inicio fué distinto a lo que normalmente estaba acostumbrado, pero finalmente; en los sistemas libres al igual que los sistemas con licencia privativa; se tienen las mismas herramientas: inicio, barras de búsqueda, explorador de archivos y una terminal. Aunque había escuchado de las libertades del software libre, no las comprendí hasta que empecé a utilizarlo, y ¿cuáles son? Se estarán preguntando. Éstas libertades son la piedra angular sobre el desarrollo de software libre.\\
 
 La primera es  la libertad de usar el programa, con cualquier propósito. Esto es totalmente contrario a los contratos de licencia que no todos acabamos de entender, en esos en dónde los usuarios exponemos nuestra privacidad a cambio de utilizar la aplicación deseada. 
 
La segunda es la libertad de estudiar cómo funciona el programa y modificarlo, adaptándolo a las propias necesidades, esto es muy importante en la mejora .

La tercera es la libertad de distribuir copias del programa, con lo cual se puede ayudar a otros usuarios.

Y la última libertad la de mejorar el programa y hacer públicas esas mejoras a los demás, de modo que toda la comunidad se beneficie.
 
 
 %La mayoría de los programas con licencia te dejan trabajar con su licencia estudiantil, o te dejan usar sus versiones en línea, entonces utilizarlos no se vuelve una complicación.\\
 
 
 
 



\end{document}